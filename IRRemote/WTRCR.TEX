                     IR Remote Control Receiver

     Have you ever wanted to make good use of all those remote
control units from TVs, VCRs, stereo equipment, etc., which have been
stacking up over the years? Have you ever wanted to add remote
operation to your electronic projects? This, simple-to-build low cost
construction project, will receive and convert the output of
virtually any infrared remote control transmitter - using a 40KHz
carrier - to logic levels which can be used to control all your
favorite toys, from robotics to rail roads. This basic "building
block" can even be used to turn on and off just about anything in
your home (lamps, fans, radios, alarms, electric locks, space
heaters, air conditioners) without leaving the comfort of your own
lounge chair. Anything which operates on electricity, can be
controlled with that $3.00 remote control - found in abundance at
surplus dealers and ham fests. The only limit being that of your
imagination!

     The IR Remote Control Receiver as explained here, has 7
individual TTL level outputs which can be programmed to correspond to
any button on your remote control and be set up for either: latching
outputs which toggle between "high" and "low" each time the button is
pressed, or momentary outputs which switch and remain "high" for the
duration the remote's button is pressed. Programming is a simple
procedure of placing the unit in the programming mode, aiming your
remote control transmitter at the project and pressing buttons to let
it "learn" and record the data streams transmitted by your particular
remote.

About Remote Control Transmitters

     The standard infrared remote control transmitter used to control
your TV, stereo, VCR, etc., uses a photo diode which transmits in the
near infrared range and is pulsed on and off at a 40KHz rate.
Although there are transmitters which use a different carrier
frequency, 40KHz is the most common, and will be used as the standard
in this circuit. In fact, upon obtaining a supply of remote controls
of various different brand names for research on this subject, the
author was unable to find one which used a frequency other then
40KHz. If, however, you have a remote control transmitter which uses
a different carrier frequency, simply substitute the 40KHz IR module
used in this circuit with one that is tuned to your transmitter's
frequency. Digi-Key has IR modules tuned to 32KHz and 36KHz and can
be reached at 800-344-4539.

     The IR signal is modulated by being transmitting in bursts of
40KHz pulses as shown in figure 1. By alternating the exact length of
the burst, or the time between bursts, it is possible to encode
intelligent data on the IR signal; this being the method used by
these remote control transmitters to indicate which button is being
pressed. Figure 1a shows an IR signal from a transmitter which
encodes the digital data stream on the carrier by alternating the
length of the burst, while figure 1b shows one which alternates the
time between bursts. A typical receiver in a host product will for
instance, decode the individual logic levels in the data stream by
comparing this pattern of bursts with an internal clock operating at
the same frequency as one within the transmitter. Although the
frequency of this clock used for timing, varies among remote control
manufacturers, as does the choice of alternating the length of the
burst or that of the time between bursts, the same basic principle
applies and can be utilized.

     In most cases the IR signal consists of a pattern of anywhere
from 12 to 32 bursts of 40KHz infrared. This pattern is continuously
repeated while the transmitter's button is held down. The author did
run across one model whose burst pattern was transmitted only upon
initial contact with the button, and was followed by a short burst
every 100ms or so, until the button was released.

Circuit Theory

     The circuit is made simple by the self contained infrared
Receiver/Demodulator (MOD1). A block diagram of the IR module is
shown in figure 2. The modulated IR signal is detected by the photo
diode which has its peak sensitivity in the near infrared range.
After passing through a preamplifier/limiter, the built-in band pass
filter then rejects all signals outside the pass band of 40KHz. This
reduces or eliminates false operation caused by other light sources.
The remaining signal is fed to the demodulator, integrator, and
comparator which outputs a clean TTL level pulse stream without the
carrier. Figure 3 demonstrates the output pulse of the IR module with
respect to the remote control transmitter's signal being received at
its input. Note, how the presence of an IR burst (figure 3a) produces
a low at the output of the IR module (figure 3b).

     Refer to the schematic diagram shown in figure 4. The heart of
the circuit is IC1 (part no. PIC16C54), an 8-bit CMOS microcontroller
manufactured by Microchip. This microcontroller has one 8-bit I/O
port and one 4-bit I/O port. Because each I/O pin can be used and
configured individually, it was possible to arrange them to greatly
simplify PC board layout and require only a single-sided board. This
is the reason for the non-uniform use of I/O pins as shown in the
schematic.

     IC1 is linked to and stores all its data in IC2 (part no.
93LC46), a 1K serial EEPROM (Electrically Erasable Programmable Read
Only Memory) also manufactured by Microchip. In this application the
93LC46 uses a 3-wire interface. Of the three wires there is a CHIP
SELECT, CLOCK, and DATA IN/OUT. Because the DATA IN and DATA OUT
share the same line, a 1K resistor (R2) is used to limit the current
flow during those transition times between WRITE and READ when there
are conflicting logic levels.

     IC1 communicates with the 93LC46 by placing a high on the CHIP
SELECT pin. Data is then transferred serially to and from the 93LC46
on the positive transition of the clock pin. Each READ or WRITE
function is preceded by a start bit, an opcode - identifying the
function to be performed (read, write, etc.) - then a 7-bit address,
followed by the 8 bits of data which is being written to, or read
from that address. Immediately preceding and following all WRITE
operations, the microcontroller sends instructions to the 93LC46
which enables/disables the WRITE function, thereby protecting the
data thereafter.

     During the programming mode, IC1 reads the IR data streams from
MOD1 and converts them to data patterns which can be stored in IC2.
These data patterns are used for comparison while in normal
operation. More on this later.

     Power for the circuit is conditioned by IC3, a low current
5-volt regulator which will accept any DC voltage between 7 and 25
volts. C1 and C2 stabilize the operation of the regulator. Crystal
(XTAL1) sets the internal oscillator of IC1 to 4MHz. JMP1 is actually
two closely spaced pads on the board which, when momentarily jumpered
with a screwdriver, will place IC1 in the programming mode and light
LED1.

The PIC Firmware

     A pre-programmed PIC16C54 is available from the supplier
mentioned in the parts list. The source and object code are available
on the Gernsback BBS for those who have the proper equipment and wish
to program their own PIC.

     As mentioned earlier, the exact protocol used to indicate the
different logic levels in the data stream transmitted by a typical IR
remote control varies from manufacturer to manufacturer. Because of
this, when recording the data streams related to each button on the
remote, the firmware in IC1 is configured NOT to try to identify
logic "1"s or "0"s, but instead, to measure the width of each IR
burst and the time between bursts; storing these values in memory.
This pattern can then be used to find a match while in normal
operation.

     When measuring, unfortunately there is only 16 8-bit registers
available in IC1 to process and hold these values before sending them
to memory. Since both the bursts and time between bursts must be
measured, a 32 burst pattern will require 64 measurements. To
compress the data so it can be handled by 16 registers, the program
in IC1 does a series of tricks as explained: Because a change in the
length of either, the burst, or the time between bursts will be used
by the transmitter to encode the data but not both, measurements of
each (per cycle) can be added together and placed in the same
register. In addition, because we are only concerned with a change in
length rather then the actual length, the most significant 4-bits of
each measured value is not important and can be stripped off. These
two processes make it possible to store 4 time values in each 8-bit
register for a total of 64 measurements.

     After being placed in the programming mode by detecting a low on
pin 5 of port-B, IC1 waits for the presence of a IR signal - pressing
of a button on the remote control transmitter. A set of 32 bursts are
sampled from the beginning of the IR signal's data stream, measured
and stored in IC2; this pattern being assigned to I/O pin #1. IC1
then flashes the LED to indicate that the recording process is
complete for that button. Depending on the duration in which the
remote control's button is held down following the flash of the LED,
the I/O pin is configured for either "toggle" or "momentary"
operation. IC1 waits for the next button to be pressed on the remote
control transmitter and repeats the above procedure; assigning this
next pattern to I/O pin #2, and so forth. After 7 patterns have been
received and stored, IC1 turns off the LED and returns to normal
operation.

     During normal operation, IC1 waits for any IR signal to be
received, determines its burst pattern in the same way as in the
programming mode, and looks for a match in memory. If a match is
found, IC1 either toggles the state of the linked I/O pin, or holds
the pin high until the IR signal ceases; depending on the
configuration of the individual I/O pin as defined in the programming
mode.

Construction

     All components fit nicely on a PC board measuring just less then
2" X 3", including an area which can be used for added components
needed in your particular application. The artwork is provided for
those who wish to make their own boards, or a pre-fabricated board
may be purchased from the source mentioned in the parts list.

     Refer to the parts placement diagram shown in figure 5 and begin
by soldering in the two IC sockets used for IC1 and IC2. Next, mount
all resistors and capacitors, paying particular attention to the
orientation of the polarized capacitor C2. Solder in the crystal
(XTAL1) and the voltage regulator (IC3). The LED should be mounted
with the long lead going to the terminal labeled "A" and the short
lead to "K". Finish by mounting the IR module; including the
soldering of the two tabs for a good ground connection to the case.

     After all components have been mounted, closely examine the
solder side of the PC board for solder bridges and/or cold solder
joints and re-solder if necessary. Carefully plug IC1 and IC2 into
their sockets using the orientation as shown in the parts placement
diagram.

Operation

     To test the Remote Control Receiver, use the setup shown in
figure 6. This circuit will turn on a separate LED for each I/O pin
which goes high; easily letting you check out the unit and get a feel
for how it works. Until an exact application and circuit has been
determined, temporarily solder 7 solid conductor wires to the I/O
terminals on the PC board and run them to a solderless breadboard to
be used for your test circuit. The solderless breadboard will allow
you to make experimental changes without having to solder and
de-solder.

     Place fresh batteries in your remote control transmitter and
setup the receiver so that you can aim the transmitter directly at
the receiver's IR module at a distance of 2 to 3 feet. DO NOT hold
the transmitter closer then 2 feet or distortion caused by
overdriving the IR module will affect the data being recorded. Also,
make sure that there is no fluorescent lights shining directly on the
IR module of your receiver. Some of these lights use a pulse
frequency close to 40KHz and will interfere with the critical
"recording" operation, causing corrupted data to be inadvertently
stored in memory.

     Apply power to the Remote Control Receiver. Locate the two
square pads directly underneath the IR module and, using a small
screwdriver, briefly short them together; this will cause the LED to
light and remain on. Point the transmitter at the receiver and press
and hold the button you wish to assign to I/O pin #1. After
approximately 1/2 second the LED will flash off then back on. If you
want I/O pin #1 to be configured as "toggle", immediately release the
button on the transmitter. If you want it to be configured as
"momentary", continue to hold the button down until the LED flashes a
second time. Next, choose the button you wish to assign to I/O pin #2
and repeat the above procedure. Continue until all 7 I/O pins have
been dedicated to a button on your transmitter. After the last I/O
pin has been programmed, the Remote Control Receiver will
automatically end the programming mode and turn off the LED. Note,
you must continue through and program all I/O pins before the
programming mode will be terminated.

     Now press those buttons on the remote control transmitter and
watch as the appropriate LED turns on and off in your test circuit.
You can assign more then one I/O pin to a single button on the
transmitter. For instance, one I/O pin could be setup to toggle
between high and low each time the button is pressed, while another
(set to momentary) could indicate how long that same button is held
down.

     figure 7 shows an I/O pin being used to drive a DC load of up to
500mA. This circuit can be repeated on all I/O pins to control 7
different loads. Applications which come to mind include controlling
servos and motors in robotics, or turning on any 9-volt battery
powered device. If larger loads are required, figure 8 shows an
interface to a relay. Although a 12V relay is shown, any relay
operating on a voltage from 7 to 25VDC can be used simply by using
its specific voltage to power the Remote Control Receiver.

     For AC applications, figure 9 uses an optoisolator and triac to
switch the line current coming from a standard 120VAC wall outlet.
This circuit will control just about anything which plugs into an AC
outlet. CAUTION! Great care should be observed when working with
120VAC. Note that most triacs have their metal tabs tied directly to
one of the main terminals, hence, YOU WILL BE SHOCKED if you touch
this tab while 120VAC is applied. If large loads are being used which
require mounting of the triac to a heat sink, use insulated mounting
hardware such as Radio Shack #276-1373 and always check for shorts to
ground with an ohmmeter prior to plugging into an AC outlet.

     For you model railroad buffs, figure 10 shows a circuit which
will control the track switches. Two I/O pins are required per switch
- one for each direction - and they must be configured as momentary.
The Remote Control Receiver can also be used to control all the other
accessories in your setup.

     The preceding is just a small example of some of the
applications which come to mind, and typical circuits which can be
used. Slight alterations may be required in your particular setup.
The Remote Control Receiver described here is a low-cost, very easy
to assemble, starting block for all your projects which you wish to
incorporate infrared remote operation. After building and
experimenting with this project, you'll find that the possibilities
are unlimited. 


Parts List

Resistors (All are 1/4-watt, 10% units)
R1 - 620 ohm
R2 - 1,000 ohm

Capacitors
C1, C5, C6 - 0.1 uF, Mylar
C2 - 10 uF, 35-WVDC, electrolytic
C3, C4 - 15 pF, ceramic disc

Semiconductors
IC1 - PIC16C54-XT/P (Pre-programmed) 8-bit microcontroller
(Microchip)
IC2 - 93LC46 serial EEPROM (Microchip)
IC3 - 78L05 low power 5-volt regulator
LED1 - light emitting diode

Other components
MOD1 - 40-KHz Infrared Remote Control Receiver Module (Digi-Key part
no. LT1060-ND or equivalent)
XTAL1 - 4 MHz crystal



The following items are available from Weeder Technologies, PO Box
421, Batavia, OH 45103. 513-752-0279.

  An etched and drilled PC board (WTRCR-B), $8.50.
  
  All Board mounting components including pre-programmed PIC16C54
  (WTRCR-C), $23.50.
  
  A pre-programmed PIC16C54 only (PIC-RCR), $16.00
  
All orders must include an additional $3.50 for shipping and
handling. Ohio residents add 6% sales tax.


Captions:

Figure 1
  The IR signal is digitally modulated by being transmitted in
  bursts. Fig 1a shows one being modulated by alternating the length
  of the burst. Fig 1b shows one which is modulated by alternating
  the time between bursts.

Figure 2
  Block diagram of the Infrared Receiver/Demodulator module (MOD1).
  
Figure 3
  The output waveform at pin 1 of the IR module with respect to a
  valid 40KHz signal being received at its input.
  
Figure 4
  Schematic diagram of the IR Remote Control Receiver. The
  microcontroller (IC1) analyzes the decoded infrared data stream
  from the IR module (MOD1), compares it to the previously stored
  patterns in the EEPROM (IC2), and takes action on the appropriate
  I/O pin if a match is found.
  
Figure 5
  Use this parts placement diagram when assembling the board.
  
Figure 6
  Use this test setup to check out your finished project. A separate
  LED indicates when each I/O pin goes high.
  
Figure 7
  This setup can be used to turn on any DC load of up to 500mA.
  
Figure 8
  For larger currents, this circuit shows the Remote Control Receiver
  interfaced to a relay.
  
Figure 9
  AC loads such as lamps, fans, stereos, etc. can be controlled using
  this setup.
  
Figure 10
  This setup can be used on a model railroad to control the track
  switches.